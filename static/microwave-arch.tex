%% ================================================================================
%% This LaTeX file was created by AbiWord.                                         
%% AbiWord is a free, Open Source word processor.                                  
%% More information about AbiWord is available at http://www.abisource.com/        
%% ================================================================================

\documentclass[a4paper,portrait,12pt]{article}
\usepackage[latin1]{inputenc}
\usepackage{calc}
\usepackage{setspace}
\usepackage{fixltx2e}
\usepackage{graphicx}
\usepackage{multicol}
\usepackage[normalem]{ulem}
%% Please revise the following command, if your babel
%% package does not support en-US
\usepackage[en]{babel}
\usepackage{color}
\usepackage{hyperref}
 
\begin{document}


\begin{flushleft}
Architecture for a microwave trapped ion quantum
\end{flushleft}


\begin{flushleft}
computer
\end{flushleft}


\begin{flushleft}
Victory Omole
\end{flushleft}


\begin{flushleft}
Iowa State University
\end{flushleft}


\begin{flushleft}
104 Marston Hall 533 Morrill Rd.
\end{flushleft}


\begin{flushleft}
Ames, IA 50011-2151
\end{flushleft}





\begin{flushleft}
vtomole@iastate.edu
\end{flushleft}





\begin{flushleft}
ABSTRACT
\end{flushleft}


\begin{flushleft}
Quantum computing is a revolutionary field that has changed
\end{flushleft}


\begin{flushleft}
the way we think about the theory of computation. Theoretically, these devices can solve certain problems that would
\end{flushleft}


\begin{flushleft}
be intractable with the classical computers we use today.
\end{flushleft}


\begin{flushleft}
Although physical implementations of quantum computers
\end{flushleft}


\begin{flushleft}
have proved to be extremely difficult, researchers have been
\end{flushleft}


\begin{flushleft}
making slow but noticeable progress in the last two decades.
\end{flushleft}


\begin{flushleft}
Over the years, there have been plans to build large scale
\end{flushleft}


\begin{flushleft}
quantum computers that would be able to scale from tens to
\end{flushleft}


\begin{flushleft}
thousands, and even millions of qubits. There have also been
\end{flushleft}


\begin{flushleft}
a countless number of architecture designs that would combine these physical implementations with implementationdependent instruction set architectures. One such design
\end{flushleft}


\begin{flushleft}
for a large scale quantum computer is the microwave ion
\end{flushleft}


\begin{flushleft}
trap quantum computer. An instruction set architecture
\end{flushleft}


\begin{flushleft}
(ISA) for this design does not exist. In this paper I will
\end{flushleft}


\begin{flushleft}
propose an ISA for this type of quantum computer. One
\end{flushleft}


\begin{flushleft}
that combines the physical implementation of a trapped
\end{flushleft}


\begin{flushleft}
ion microwave quantum computer with an ISA designed
\end{flushleft}


\begin{flushleft}
by Rigetti Computing. This architecture will contain two
\end{flushleft}


\begin{flushleft}
compilers. The first will transform the implementationindependent ISA to an implementation dependent one. The
\end{flushleft}


\begin{flushleft}
second will be an error correcting compiler that will transform an implementation-independent ISA to an implementation dependent error correcting ISA.
\end{flushleft}





\begin{flushleft}
CCS Concepts
\end{flushleft}


\begin{flushleft}
$\bullet$Computer systems organizations $\rightarrow$ Quantum computing;
\end{flushleft}





\begin{flushleft}
Keywords
\end{flushleft}


\begin{flushleft}
Computer architecture, Quantum computing, Compiler
\end{flushleft}





1.





\begin{flushleft}
INTRODUCTION
\end{flushleft}





\begin{flushleft}
Quantum computers can solve certain problems; such as
\end{flushleft}


\begin{flushleft}
factoring and simulation of quantum systems, which can-
\end{flushleft}





\begin{flushleft}
Permission to make digital or hard copies of part or all of this work for personal or
\end{flushleft}


\begin{flushleft}
classroom use is granted without fee provided that copies are not made or distributed
\end{flushleft}


\begin{flushleft}
for profit or commercial advantage and that copies bear this notice and the full citation
\end{flushleft}


\begin{flushleft}
on the first page. Copyrights for third-party components of this work must be honored.
\end{flushleft}


\begin{flushleft}
For all other uses, contact the owner/author(s).
\end{flushleft}


\begin{flushleft}
c 2017 Copyright held by the owner/author(s).
\end{flushleft}








\begin{flushleft}
not be solved efficiently on classical computers. Quantum
\end{flushleft}


\begin{flushleft}
computers can be generalized into two categories. Nearterm quantum computers, and long term-quantum computers. Near-term quantum computers are computers that will
\end{flushleft}


\begin{flushleft}
be realized in the next 5 to 10 years. They are composed
\end{flushleft}


\begin{flushleft}
of tens to hundreds of qubits. Long term quantum computers could potentially be realized in a few decades. They
\end{flushleft}


\begin{flushleft}
will contain thousands to millions of qubits. This paper will
\end{flushleft}


\begin{flushleft}
address both types of computers.
\end{flushleft}


\begin{flushleft}
Researchers at the Department of Physics and Astronomy at the University of Sussex have recently published a
\end{flushleft}


\begin{flushleft}
blueprint for a large scale trapped ion quantum computer
\end{flushleft}


\begin{flushleft}
[5]. This computer can serve as a near-term and long-term
\end{flushleft}


\begin{flushleft}
quantum computer because it is modular. The only factor
\end{flushleft}


\begin{flushleft}
that depends on whether or not it is a near-term or long-term
\end{flushleft}


\begin{flushleft}
computer, is how many individual modules can be added to
\end{flushleft}


\begin{flushleft}
the hardware before the quality of the qubits worsen to the
\end{flushleft}


\begin{flushleft}
point where the computer cannot execute its instructions
\end{flushleft}


\begin{flushleft}
correctly.
\end{flushleft}


\begin{flushleft}
This paper will first cover the basics of quantum computing. It will then transition into discussing previous research
\end{flushleft}


\begin{flushleft}
that was performed in this area. This will be followed by a
\end{flushleft}


\begin{flushleft}
couple of paragraphs about a new type of architecture for
\end{flushleft}


\begin{flushleft}
a microwave ion trapped quantum computer. After talking
\end{flushleft}


\begin{flushleft}
about the benefits of the design, the last two sections will
\end{flushleft}


\begin{flushleft}
contain a concluding paragraph and a discussion about some
\end{flushleft}


\begin{flushleft}
potential directions for future research.
\end{flushleft}





2.





\begin{flushleft}
PRELIMINARIES
\end{flushleft}





2.1


2.1.1





\begin{flushleft}
Quantum Mechanics
\end{flushleft}


\begin{flushleft}
Two-level Quantum System
\end{flushleft}





\begin{flushleft}
The fundamentals of quantum theory can be illustrated by
\end{flushleft}


\begin{flushleft}
the double slit experiment [3]. This experiment is composed
\end{flushleft}


\begin{flushleft}
of an electron source, a wall with two slits, and a backdrop.
\end{flushleft}


\begin{flushleft}
When an electron is shot at the wall with slits, there is an
\end{flushleft}


\begin{flushleft}
interference pattern on the backdrop. This result suggests
\end{flushleft}


\begin{flushleft}
that electrons are waves. There is a dilemma. The electron
\end{flushleft}


\begin{flushleft}
left the source as a particle, so which slit did it go through?
\end{flushleft}


\begin{flushleft}
To answer this question, we place a detector in front of the
\end{flushleft}


\begin{flushleft}
slits. This detector detects that the electron goes through
\end{flushleft}


\begin{flushleft}
one slit or the other, not both. The electron's behavior seems
\end{flushleft}


\begin{flushleft}
to depend on whether or not its state is observed. When it's
\end{flushleft}


\begin{flushleft}
observed, it behaves like a particle, when not, it behaves
\end{flushleft}


\begin{flushleft}
like wave. The act of measuring the electron collapses the
\end{flushleft}


\begin{flushleft}
wave function. The double-slit experiment highlights the
\end{flushleft}


\begin{flushleft}
wave-particle duality of light. This behavior describes all
\end{flushleft}





\begin{flushleft}
\newpage
quantum systems, such as photons and atoms. It cannot be
\end{flushleft}


\begin{flushleft}
explained using classical physics.
\end{flushleft}


\begin{flushleft}
The consequences of the double-slit experiment can also
\end{flushleft}


\begin{flushleft}
be described as a two-level quantum system. The doubleslit experiment and the two-level quantum systems highlight
\end{flushleft}


\begin{flushleft}
that quantum particles can be in a superposition of states.
\end{flushleft}


\begin{flushleft}
For example, an atom is composed of two states. A ground
\end{flushleft}


\begin{flushleft}
state and an excited state. When the atom is not measured, it is in a superposition of the ground state and the
\end{flushleft}


\begin{flushleft}
excited state. As soon as it is measured, there is a 50 percent chance that the atom will be in a ground state and a 50
\end{flushleft}


\begin{flushleft}
percent chance that it will be in an excited state. Quantum
\end{flushleft}


\begin{flushleft}
mechanics can be described by probability theory.
\end{flushleft}





2.1.2





\begin{flushleft}
Quantum Entanglement
\end{flushleft}





\begin{flushleft}
Quantum entanglement is when quantum objects interact
\end{flushleft}


\begin{flushleft}
so that the state of each object cannot be described independently of the other objects [8]. When a pair of quantum
\end{flushleft}


\begin{flushleft}
particles are entangled, it is possible to know the state of
\end{flushleft}


\begin{flushleft}
both particles by measuring one of them. For example, if
\end{flushleft}


\begin{flushleft}
two electrons are entangled, they can be described as being
\end{flushleft}


\begin{flushleft}
in the superposition of states spin up and spin down. If one
\end{flushleft}


\begin{flushleft}
electron is measured, then the state of that electron is correlated with the state of the other. If the measured electron
\end{flushleft}


\begin{flushleft}
is spin up, then the other one is also spin up.
\end{flushleft}





2.1.3





2.2.1





2.2.2





\begin{flushleft}
Quantum gates
\end{flushleft}





\begin{flushleft}
Like classical computing, quantum computing has finite
\end{flushleft}


\begin{flushleft}
set of universal gates. A set of universal quantum gates can
\end{flushleft}


\begin{flushleft}
be used to construct any arbitrary quantum algorithm. The
\end{flushleft}


\begin{flushleft}
most common types of universal quantum gates are onequbit gates and two-qubit gates. One-qubit gates can be
\end{flushleft}


\begin{flushleft}
represented with 2 by 2 matrices. They transform the state
\end{flushleft}


\begin{flushleft}
vector of one qubit. 4 by 4 matrices are used to represent two
\end{flushleft}


\begin{flushleft}
qubit gates. They are used to entangle 2 qubits with each
\end{flushleft}


\begin{flushleft}
other. A set of universal quantum gates can be composed
\end{flushleft}


\begin{flushleft}
of Pauli and CNOT gates. They can also be composed of
\end{flushleft}


\begin{flushleft}
Hadamard, Phase shift, and CNOT gates [6].
\end{flushleft}


?


\begin{flushleft}
P auli $-$ X =
\end{flushleft}





0


1





?


1


0





\begin{flushleft}
P auli $-$ Y =
\end{flushleft}


\begin{flushleft}

\end{flushleft}





?


?


1 0


\begin{flushleft}
P auli $-$ Z =
\end{flushleft}


0 $-$1





\begin{flushleft}
Decoherence
\end{flushleft}





\begin{flushleft}
A system maintains its quantum state as long as the environment does not disturb it. All quantum systems lose their
\end{flushleft}


\begin{flushleft}
quantum states because they can never be perfectly isolated
\end{flushleft}


\begin{flushleft}
from their environments. Decoherence is when a quantum
\end{flushleft}


\begin{flushleft}
state is disrupted by the environment. When two-level quantum systems decohere, they go from being in a superposition
\end{flushleft}


\begin{flushleft}
of states to one state. For example, an electron can be in
\end{flushleft}


\begin{flushleft}
a superposition of spin up and spin down states for a short
\end{flushleft}


\begin{flushleft}
period of time before decohering to only being in a spin up
\end{flushleft}


\begin{flushleft}
or spin down state. The length of time it takes for a particle
\end{flushleft}


\begin{flushleft}
to lose its superposition of states is called the decoherence
\end{flushleft}


\begin{flushleft}
time.
\end{flushleft}





2.2





\begin{flushleft}
quantum system. This means that is can be represented as
\end{flushleft}


\begin{flushleft}
a vector on two dimensional complex vector space, like a
\end{flushleft}


\begin{flushleft}
Bloch sphere. Computations are performed by rotating the
\end{flushleft}


\begin{flushleft}
state vector in the sphere before the qubit decoheres [6].
\end{flushleft}





?


?


1


0


\begin{flushleft}
$\pi$/8 =
\end{flushleft}


\begin{flushleft}
0 e$\pi$/4i
\end{flushleft}





2.2.3





1


\begin{flushleft}
0
\end{flushleft}


\begin{flushleft}
CN OT = 
\end{flushleft}


\begin{flushleft}
0
\end{flushleft}


0


?





?


?


\begin{flushleft}
0 $-$i
\end{flushleft}


\begin{flushleft}
i 0
\end{flushleft}


0


1


0


0





\begin{flushleft}

\end{flushleft}


0


\begin{flushleft}
0
\end{flushleft}


\begin{flushleft}

\end{flushleft}


\begin{flushleft}
1
\end{flushleft}


0





0


0


0


1





1 1


1 $-$1


$\surd$


\begin{flushleft}
Hadamard =
\end{flushleft}


2





?





\begin{flushleft}
Quantum Algorithms
\end{flushleft}





\begin{flushleft}
A quantum algorithm is the application of successive quantum gates to qubits. It can be represented as multiplying the
\end{flushleft}


\begin{flushleft}
vector that represents a gate with the vector that represents
\end{flushleft}


\begin{flushleft}
a qubit. It can also be expressed as a circuit diagram.
\end{flushleft}





\begin{flushleft}
Quantum Computing
\end{flushleft}


\begin{flushleft}
Quantum bit
\end{flushleft}





\begin{flushleft}
A fundamental unit of information for classical computers
\end{flushleft}


\begin{flushleft}
is the bit. The fundamental unit for quantum computers is
\end{flushleft}


\begin{flushleft}
the quantum bit, or qubit. It can be represented with a
\end{flushleft}


\begin{flushleft}
Bloch sphere.
\end{flushleft}


\begin{flushleft}
ẑ = |0i
\end{flushleft}


\begin{flushleft}
|$\psi$i
\end{flushleft}





\begin{flushleft}
The above circuit diagram is the Bell state algorithm.
\end{flushleft}


\begin{flushleft}
This algorithm is an illustration of quantum entanglement.
\end{flushleft}


\begin{flushleft}
The Hadamard gate is applied to the first qubit, which transforms the state of the first qubit to a superposition. The
\end{flushleft}


\begin{flushleft}
CNOT gate is applied from the first qubit to the second
\end{flushleft}


\begin{flushleft}
one. This will perform the NOT operation on the second
\end{flushleft}


\begin{flushleft}
qubit if the first one is in an excited state, and do nothing
\end{flushleft}


\begin{flushleft}
if the first qubit is in a ground state [6].
\end{flushleft}





2.2.4


\begin{flushleft}
$\theta$
\end{flushleft}


\begin{flushleft}
ŷ
\end{flushleft}


\begin{flushleft}
ϕ
\end{flushleft}


\begin{flushleft}
x̂
\end{flushleft}





2.2.5


\begin{flushleft}
$-$ẑ = |1i
\end{flushleft}


\begin{flushleft}
While a bit can be represented as a vector that occupies
\end{flushleft}


\begin{flushleft}
one of two states, a qubit needs to be described as a two-level
\end{flushleft}





\begin{flushleft}
Quantum error Correction
\end{flushleft}





\begin{flushleft}
Quantum error correction techniques are used when the
\end{flushleft}


\begin{flushleft}
quantum algorithms need to be protected from decoherence
\end{flushleft}


\begin{flushleft}
and noise; because decoherence and noise destroys the computation. Redundant qubits are used in the physical implementation of a quantum computer to preserve the state of
\end{flushleft}


\begin{flushleft}
the computation for as long as possible [6].
\end{flushleft}





\begin{flushleft}
DiVincenzo's criteria
\end{flushleft}





\begin{flushleft}
DiVincenzo's criteria is a list of specifications created by
\end{flushleft}


\begin{flushleft}
David DiVincenzo that a physical system must satisfy to be
\end{flushleft}


\begin{flushleft}
considered a universal quantum computer [2].
\end{flushleft}





\begin{flushleft}
\newpage
1. System of qubits
\end{flushleft}


\begin{flushleft}
2. Method of initializing the qubits to a ground state
\end{flushleft}


\begin{flushleft}
3. Qubits with long decoherence times
\end{flushleft}


\begin{flushleft}
4. Implementation of a universal set of quantum gates
\end{flushleft}


\begin{flushleft}
5. Method of measuring state of qubits
\end{flushleft}





2.2.6





\begin{flushleft}
Computation with Ion-traps
\end{flushleft}





\begin{flushleft}
Ion traps can be used to meet DiVincenzo's criteria. They
\end{flushleft}


\begin{flushleft}
are devices that use magnetic and electric fields to trap ions
\end{flushleft}


\begin{flushleft}
[8]. Each ion that is trapped represents a qubit. These
\end{flushleft}


\begin{flushleft}
qubits are initialized by cooling them to their ground states.
\end{flushleft}


\begin{flushleft}
Ions are a great representation for qubits because they have
\end{flushleft}


\begin{flushleft}
long decoherence times (from hundreds to thousands of microseconds) [6].
\end{flushleft}


\begin{flushleft}
Universal quantum gates are usually implemented by shining a global laser on the ions for multiple qubit gates and
\end{flushleft}


\begin{flushleft}
focused lasers for single qubit gates. The state of the qubit
\end{flushleft}


\begin{flushleft}
is measured by shining a laser on an ion and observing if the
\end{flushleft}


\begin{flushleft}
ion emits a photon. If the ion emits a photon, then the ion
\end{flushleft}


\begin{flushleft}
is in an excited state. If nothing is emitted, then the ion is
\end{flushleft}


\begin{flushleft}
in a ground state. The frequency for the laser that performs
\end{flushleft}


\begin{flushleft}
measurements and the one that applies quantum gates is different, because ions need to maintain their quantum state
\end{flushleft}


\begin{flushleft}
after the gates are applied.
\end{flushleft}





3.





\begin{flushleft}
PREVIOUS WORK
\end{flushleft}





\begin{flushleft}
Most computer architectures whether classical or quantum, generally follow the same design. The architecture is
\end{flushleft}


\begin{flushleft}
composed of an instruction set. This instruction set describes a language for an abstract machine which could implemented in software or hardware.
\end{flushleft}





3.1


3.1.1





\begin{flushleft}
Laser ion trapped architecture
\end{flushleft}


\begin{flushleft}
Instruction Set Architecture
\end{flushleft}





\begin{flushleft}
Except for some minor differences, most ISAs for quantum
\end{flushleft}


\begin{flushleft}
computers are the same. The high level ISA is represented
\end{flushleft}


\begin{flushleft}
as the text representation of the quantum gates. An ISA for
\end{flushleft}


\begin{flushleft}
a laser trapped ion computer could be the based on the Von
\end{flushleft}


\begin{flushleft}
Neumann model [5].
\end{flushleft}





3.1.2





\begin{flushleft}
Physical implementation
\end{flushleft}





\begin{flushleft}
This computer is composed of several ion traps that are
\end{flushleft}


\begin{flushleft}
connected with wires. Computations are performed by applying lasers on the ions. The wires are used to transport
\end{flushleft}


\begin{flushleft}
ions between traps.
\end{flushleft}





3.2





\begin{flushleft}
Quil
\end{flushleft}





\begin{flushleft}
A hybrid quantum computer is a computer that treats a
\end{flushleft}


\begin{flushleft}
quantum computer as a co-processor to a classical computer.
\end{flushleft}


\begin{flushleft}
It does not give the quantum computer a job that cannot be
\end{flushleft}


\begin{flushleft}
performed efficiently on the classical computer. Near-term
\end{flushleft}


\begin{flushleft}
quantum computers will be hybrid. One instruction set for
\end{flushleft}


\begin{flushleft}
hybrid quantum computers is Quil. This instruction language describes an abstract machine for a hybrid quantum
\end{flushleft}


\begin{flushleft}
computer. It is primarily designed for near-term algorithms;
\end{flushleft}


\begin{flushleft}
like the variational quantum eigensolver and material simulations [7].
\end{flushleft}





3.3





\begin{flushleft}
Microwave ion trap quantum computer
\end{flushleft}





3.3.1





\begin{flushleft}
Physical implementation
\end{flushleft}





\begin{flushleft}
The trapped ion microwave quantum computer is made
\end{flushleft}


\begin{flushleft}
out of individual modules. Each module is a universal quantum computer. It contains a loading zone, gate zone and
\end{flushleft}


\begin{flushleft}
readout zone. The loading zone is where ions are initialized. Quantum gates are applied to the ions in the gate
\end{flushleft}


\begin{flushleft}
zone. Instead of using lasers to perform the gates like the
\end{flushleft}


\begin{flushleft}
laser trapped ion computer, this computer performs gates
\end{flushleft}


\begin{flushleft}
using microwave radiation. The readout zone is where a
\end{flushleft}


\begin{flushleft}
measurement laser is applied to the ion to read the result of
\end{flushleft}


\begin{flushleft}
the computation. Individual modules are attached to each
\end{flushleft}


\begin{flushleft}
other to create a large scale quantum computer [5].
\end{flushleft}





3.3.2





\begin{flushleft}
Error correction
\end{flushleft}





\begin{flushleft}
The trapped ion computer uses surface codes to perform
\end{flushleft}


\begin{flushleft}
error correction [5]. The physical qubits are separated into
\end{flushleft}


\begin{flushleft}
data qubits, measure-x and measure-z qubits. The measurez qubit applies 4 CNOT gates on the data qubit. The
\end{flushleft}


\begin{flushleft}
measure-x performs the Hadamard gate before and after the
\end{flushleft}


\begin{flushleft}
4 CNOT gates have been applied. It is designed so that if
\end{flushleft}


\begin{flushleft}
there is an error in a data bit, it flips the neighboring operators that will detect the error. The data for these errors will
\end{flushleft}


\begin{flushleft}
be transferred to a classical computer, where a classical program will correct the results of the quantum computation
\end{flushleft}


\begin{flushleft}
based on the error information.
\end{flushleft}





4.


4.1





\begin{flushleft}
NEW ARCHITECTURE
\end{flushleft}


\begin{flushleft}
Quil compiler
\end{flushleft}





\begin{flushleft}
I propose a new architecture that will implement the Quil
\end{flushleft}


\begin{flushleft}
ISA on the microwave ion trapped quantum computer. A
\end{flushleft}


\begin{flushleft}
lot of architectures have been proposed for ion trap quantum computers. This design borrows ideas from Balensiefer,
\end{flushleft}


\begin{flushleft}
Kregor-Stickles, and Oskin's infrastructure; which is composed of a source compiler, an error correction compiler,
\end{flushleft}


\begin{flushleft}
a device scheduler, and a simulator [1]. The source compiler will generate Quil code. Since Quil code is a low-level
\end{flushleft}


\begin{flushleft}
intermediate representation of quantum programs [7], The
\end{flushleft}


\begin{flushleft}
Quil compiler will be a back-end compiler. It will contain
\end{flushleft}


\begin{flushleft}
3 phases. Program analysis, optimization, and code generation. The code generator will output an architecturedependent ISA. This architecture-dependent ISA will have
\end{flushleft}


\begin{flushleft}
a one-to-one correspondence with the machine code that will
\end{flushleft}


\begin{flushleft}
be executed on the computer.
\end{flushleft}





4.2





\begin{flushleft}
Error correcting compiler
\end{flushleft}





\begin{flushleft}
Error correction will not be needed for near-term quantum devices. When quantum computers are composed of
\end{flushleft}


\begin{flushleft}
enough qubits, we will be capable of programming them
\end{flushleft}


\begin{flushleft}
to perform error corrected computations. This is why the
\end{flushleft}


\begin{flushleft}
error correcting compiler is optional. Unlike the Quil compiler, the computer programmer can choose to incorporate
\end{flushleft}


\begin{flushleft}
the error correction compiler into his/her architecture. The
\end{flushleft}


\begin{flushleft}
error correcting compiler will be a source to source compiler. It will compile Quil code to error corrected Quil code.
\end{flushleft}


\begin{flushleft}
If the computer architect wishes to incorporate the error correction compiler into his/her architecture, it will be placed
\end{flushleft}


\begin{flushleft}
before the Quil compiler. Its input will be the Quil code
\end{flushleft}


\begin{flushleft}
that that was written by the programmer or code generated
\end{flushleft}


\begin{flushleft}
by a higher level compiler, and it's output will be the error corrected version of that program. The error corrected
\end{flushleft}


\begin{flushleft}
code will be valid Quil code. This error corrected program
\end{flushleft}


\begin{flushleft}
will be the input to the Quil compiler, which will generate
\end{flushleft}





\begin{flushleft}
\newpage
the architecture-dependent ISA. The error correction compiler will implement the surface code error correction scheme
\end{flushleft}


\begin{flushleft}
that was covered in section 3.3.2.
\end{flushleft}





5.





\begin{flushleft}
BENEFITS OF THE NEW DESIGN
\end{flushleft}





\begin{flushleft}
Quil is a low-level intermediate representation. This means
\end{flushleft}


\begin{flushleft}
that it can be treated like the LLVM Intermediate representation to build reusable compilers [4]. Compiler writers who
\end{flushleft}


\begin{flushleft}
choose Quil as a target language do not have to write a backend for their compiler. Using Quil as the implementatationindependant ISA for the microwave computer takes advantage of this reusability.
\end{flushleft}


\begin{flushleft}
The modularity of the architecture means that other architecture designers can take the parts they like about this
\end{flushleft}


\begin{flushleft}
implementation and leave parts they would like to implement on their own. If they like the error correction compiler,
\end{flushleft}


\begin{flushleft}
but their implementation targets a different back-end like a
\end{flushleft}


\begin{flushleft}
superconducting quantum computer, they can take the error correction compiler and implement everything else. They
\end{flushleft}


\begin{flushleft}
could also decide to add their backend to the Quil compiler.
\end{flushleft}





6.





\begin{flushleft}
CONCLUSION
\end{flushleft}





\begin{flushleft}
I have proposed a scalable software architecture for a scalable quantum computer. After reviewing the available tools,
\end{flushleft}


\begin{flushleft}
I have found that they were not sufficient for near-term and
\end{flushleft}


\begin{flushleft}
long-term ion trap quantum computers. The Quil ISA is
\end{flushleft}


\begin{flushleft}
designed for near-term quantum computers because it compiles straight to the implementation of an abstract machine
\end{flushleft}


\begin{flushleft}
without first going through an error correction compiler. If
\end{flushleft}


\begin{flushleft}
the Quil ISA is compiled to an error correcting ISA, it can
\end{flushleft}


\begin{flushleft}
be suitable for long-term quantum computers. The ideas
\end{flushleft}


\begin{flushleft}
presented are mainly theoretical, but there is a plan to test
\end{flushleft}


\begin{flushleft}
them in future research.
\end{flushleft}





7.





\begin{flushleft}
effects. It will also test the viability of the microwave quantum computer by testing how well the computer modules
\end{flushleft}


\begin{flushleft}
scale with respect to the number of qubits. This simulator
\end{flushleft}


\begin{flushleft}
might follow design of the Monte-Carlo simulator [1].
\end{flushleft}





\begin{flushleft}
FUTURE WORK
\end{flushleft}





\begin{flushleft}
The ideas presented in this paper are not complete; specifically, how the Quil and error correction compilers will be
\end{flushleft}


\begin{flushleft}
implemented. This paper does not go into the detail about
\end{flushleft}


\begin{flushleft}
the algorithms and data structures these compilers will use.
\end{flushleft}


\begin{flushleft}
It also doesn't talk about the design of the architecturedependent ISA. Specifications of the compilers and the architecture dependent ISA might be presented in future papers.
\end{flushleft}


\begin{flushleft}
The difference between a virtual machine and a simulator is that a virtual machine is an implementation of an
\end{flushleft}


\begin{flushleft}
abstract machine, while a simulator is a software implementation of a specific hardware. An open source simulator for
\end{flushleft}


\begin{flushleft}
near-term hybrid quantum computers does not exist. Most
\end{flushleft}


\begin{flushleft}
of the open source simulators are not practical for running
\end{flushleft}


\begin{flushleft}
near-tem quantum algorithms like the variational quantum
\end{flushleft}


\begin{flushleft}
eigensolver. If not incomplete, they are usually for pedagogical or research purposes. A possible direction for the future
\end{flushleft}


\begin{flushleft}
would be implementing an open source virtual machine for
\end{flushleft}


\begin{flushleft}
near-term devices, and a wide range of programming tools
\end{flushleft}


\begin{flushleft}
on top of this virtual machine.
\end{flushleft}


\begin{flushleft}
Another possible direction is the implementation of an
\end{flushleft}


\begin{flushleft}
open source simulator of the microwave trapped ion quantum computer. This is because there is no way to know how
\end{flushleft}


\begin{flushleft}
the sound the ideas presented in this paper are until they
\end{flushleft}


\begin{flushleft}
are implemented in software. This simulator will be used
\end{flushleft}


\begin{flushleft}
for testing purposes, like how decoherence affects the computation, and how error correcting algorithms mitigate these
\end{flushleft}





8.





\begin{flushleft}
ACKNOWLEDGMENTS
\end{flushleft}





\begin{flushleft}
I would like to thank Joseph Zambreno for providing valuable feedback on how to study this subject. I would also like
\end{flushleft}


\begin{flushleft}
to thank Lucas Skoff, whose friendship kept me sane while
\end{flushleft}


\begin{flushleft}
I was performing this research.
\end{flushleft}





9.





\begin{flushleft}
REFERENCES
\end{flushleft}





\begin{flushleft}
[1] S. Balensiefer, L. Kregor-Stickles, and M. Oskin. An
\end{flushleft}


\begin{flushleft}
evaluation framework and instruction set architecture
\end{flushleft}


\begin{flushleft}
for ion-trap based quantum micro-architectures. In
\end{flushleft}


\begin{flushleft}
Proceedings of the 32nd annual international
\end{flushleft}


\begin{flushleft}
symposium on Computer Architecture, pages 186--196,
\end{flushleft}


\begin{flushleft}
New York, NY, USA, 2005. ACM.
\end{flushleft}


\begin{flushleft}
[2] D. P. DiVincenzo. The physical implementation of
\end{flushleft}


\begin{flushleft}
quantum computation. Fortschritte der Physik,
\end{flushleft}


48(9-11):771--783, 2000.


\begin{flushleft}
[3] R. Feynman, S. Sands, Matthew, and R. Leighton. The
\end{flushleft}


\begin{flushleft}
Feynman Lectures on Physics Volume 3: QUantum
\end{flushleft}


\begin{flushleft}
Mechanics. Basic Books, 250 West 57th Street, 15th
\end{flushleft}


\begin{flushleft}
Floor, New York, NY,10107, USA, 1965.
\end{flushleft}


\begin{flushleft}
[4] C. Lattner and V. Adve. LLVM: A Compilation
\end{flushleft}


\begin{flushleft}
Framework for Lifelong Program Analysis \&
\end{flushleft}


\begin{flushleft}
Transformation. In Proceedings of the 2004
\end{flushleft}


\begin{flushleft}
International Symposium on Code Generation and
\end{flushleft}


\begin{flushleft}
Optimization (CGO'04), Palo Alto, California, Mar
\end{flushleft}


2004.


\begin{flushleft}
[5] B. Lekitsch, S. Weidt, A. G. Fowler, K. M{\o}lmer,
\end{flushleft}


\begin{flushleft}
S. J. Devitt, C. Wunderlich, and W. K. Hensinger.
\end{flushleft}


\begin{flushleft}
Blueprint for a microwave trapped ion quantum
\end{flushleft}


\begin{flushleft}
computer. Science Advances, 3, July 2017.
\end{flushleft}


\begin{flushleft}
[6] M. Nielsen and I. Chuang. Quantum Computation and
\end{flushleft}


\begin{flushleft}
Quantum Information. Cambridge University Press,
\end{flushleft}


\begin{flushleft}
New York, NY, USA, 2010.
\end{flushleft}


\begin{flushleft}
[7] R. S. Smith, M. J. Curtis, and W. J. Zeng. A practical
\end{flushleft}


\begin{flushleft}
quantum instruction set architecture, 2016.
\end{flushleft}


\begin{flushleft}
[8] P. Wieburg. A Linear Paul Trap for Ytterbium Ions.
\end{flushleft}


\begin{flushleft}
Master's thesis, Universität Hamburg, 20146 Hamburg,
\end{flushleft}


\begin{flushleft}
Germany, 2014.
\end{flushleft}





\newpage



\end{document}
